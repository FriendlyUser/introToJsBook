{\justifying
	\chapterimage{png/async.png}{3cm}
	\chapter{Asynchronous Programming}
	\epigraph{The only way to learn a new programming language is by writing programs in it}{--- \textup{Dennis Ritchie}}


 Asynchronous programming in JavaScript refers to the concept of non-blocking I/O operations. This means that when an asynchronous operation is performed, the program continues to execute the next instruction without waiting for the asynchronous operation to complete. This can be achieved using callbacks, promises, and async/await.

 \section{Introduction to Promises}

A JavaScript promise is an object that represents the eventual result of an asynchronous operation. A promise can be in one of three states: fulfilled, rejected, or pending. A fulfilled promise means that the asynchronous operation has completed successfully and a value is available. A rejected promise means that the asynchronous operation has failed and an error is available. A pending promise means that the asynchronous operation is still in progress.

Promises are a better alternative to callback functions for handling asynchronous operations in JavaScript, because they make it easier to write and maintain code that uses asynchronous operations. Promises provide a cleaner and more intuitive syntax for working with asynchronous operations, and they can be composed together to create complex asynchronous behavior.

\paragraph{Why use promises}
Callback hell is a term used to describe the problem of deeply nested callback functions in JavaScript code. This can make the code difficult to read and maintain, and can lead to problems with the execution order of the asynchronous operations. To avoid callback hell, it is recommended to use the async/await syntax introduced in ES2017, or to use promises and the Promise.then() method.

 Here is an example of using promises in JavaScript:

 \begin{lstlisting}[language=Javascript, caption=Example of promises]
 const myPromise = new Promise((resolve, reject) => {
  // do something asynchronous
  if (/* asynchronous operation was successful */) {
    resolve(/* result of the asynchronous operation */);
  } else {
    reject(/* error occurred during the asynchronous operation */);
  }
});

myPromise
  .then((result) => {
    // do something with the result of the promise
  })
  .catch((error) => {
    // handle any error that occurred during the promise
  });
 \end{lstlisting}
 In the example above, the myPromise object is created with a function that performs an asynchronous operation. The function takes two arguments, resolve and reject, which are used to signal the completion or failure of the asynchronous operation. The then() method is used to specify a callback function that is called when the promise is fulfilled (i.e., the asynchronous operation is successful), and the catch() method is used to specify a callback function that is called if the promise is rejected (i.e., an error occurred during the asynchronous operation).


 // usage of promise


% https://www.goodreads.com/quotes/tag/javascript#:~:text=%E2%80%9CJava%20is%20to%20JavaScript%20what%20Car%20is%20to%20Carpet.%E2%80%9D&text=%E2%80%9CJavaScript's%20global%20scope%20is%20like,with%20surfaces%20when%20you%20do.%E2%80%9D&text=%E2%80%9CRequireJS%20is%20a%20JavaScript%20file%20and%20module%20loader.
% add index to book

\section{Http Requests}

HTTP, or Hypertext Transfer Protocol, is a networking protocol that is used to transfer data on the web. HTTP requests are messages sent by a client, such as a web browser, to a server to request information or perform actions. The server then responds to the request with an HTTP response message.

There are several different types of HTTP requests, each of which is used for a specific purpose. The most common types of HTTP requests are:
\begin{itemize}
\item GET: A GET request is used to retrieve data from a server. This type of request is typically used to retrieve a web page or other resource from a server.

\item POST: A POST request is used to send data to a server for processing. This type of request is typically used when a user submits a form on a web page, and the data from the form is sent to the server for processing.

\item PUT: A PUT request is used to update a resource on a server. This type of request is typically used to update an existing web page or other resource on a server.

\item DELETE: A DELETE request is used to delete a resource on a server. This type of request is typically used to remove a web page or other resource from a server.
\end{itemize}

These are the most common types of HTTP requests, but there are many other types of requests that can be used for different purposes. HTTP requests are an important part of how the web works, as they allow clients and servers to communicate and exchange information.

\paragraph{What is an Endpoint}

An endpoint is a specific URL that is used to access a web service or API. An endpoint typically specifies the location of a specific resource or service on a server, and includes any necessary parameters or query string values.

For example, consider a web service that allows users to search for books by title. The endpoint for this service might be something like https://example.com/books?title=harry+potter, where https://example.com/books is the base URL for the service, and title=harry+potter is a query string parameter that specifies the search term.

In this example, the endpoint is the full URL that is used to access the book search service. When a client, such as a web browser, makes an HTTP request to this endpoint, the server responds with the search results for the specified query.

Endpoints are an important part of how web services and APIs work, as they provide a way for clients to access the specific resources or services that are offered by the server. Endpoints typically include the base URL for the service, as well as any necessary parameters or query string values, to specify the exact resource or action that is being requested.

To make an HTTP request in \indextextit{JavaScript}, you can use the \indextextit{XMLHttpRequest} object or the fetch API. Here's an example of how to use \indextextit{XMLHttpRequest} to make a GET request to fetch some data from a server:


\begin{lstlisting}[language=Javascript, caption=How to make a HTTP request in Javascript]
var xhr = new XMLHttpRequest();
xhr.open('GET', 'https://www.example.com/api/data', true);

xhr.onload = function() {
  if (this.status == 200) {
    var data = JSON.parse(this.response);
    // do something with the data
  }
};

xhr.send();

\end{lstlisting}

Here's an example of how to use the fetch API to make the same request:

\begin{lstlisting}[language=Javascript, caption=HTTP request using fetch]
fetch('https://www.example.com/api/data')
  .then(response => response.json())
  .then(data => {
    // do something with the data
  });

\end{lstlisting}

Both \indextextit{XMLHttpRequest} and fetch allow you to specify additional options such as the request headers, and you can use them to make other types of HTTP requests such as POST, PUT, and DELETE.


\subsection{Using Fetch}
In JavaScript, the fetch() method is used to perform HTTP requests. It is a modern way to make network requests to retrieve resources from a server. fetch() is similar to other web request APIs like XMLHttpRequest (XHR).

The PokeAPI is a free and open-source API for accessing data about the Pokémon video game series. The API provides a GraphQL endpoint that allows you to query the API using the GraphQL language.

Here is an example of using the fetch() method to retrieve data about a Pokemon from the PokeAPI:

\begin{lstlisting}[language=Javascript, caption=Fetching entry from poke api]
fetch('https://pokeapi.co/api/v2/pokemon/1')
  .then(response => response.json())
  .then(data => {
    console.log(data);
    // do something with the data here
  });
\end{lstlisting}

In this example, the fetch() method is used to make a GET request to the PokeAPI to retrieve information about the Pokemon with the ID of 1, which is Bulbasaur. The response.json() method is used to parse the response as JSON, and then the data is logged to the console.


To use authentication headers with the fetch function in JavaScript, you can pass an object with the headers property as the second argument to the fetch function. The headers property should be an object that contains the key-value pairs for the headers you want to include in the request.

For example, if you wanted to include an Authorization header with a bearer token, you could do something like this:

\begin{lstlisting}[language=Javascript, caption="Authroization header example with fetch]
const headers = {
  'Authorization': 'Bearer <your-bearer-token-here>'
};

fetch('https://example.com/api/v1/data', { headers })
  .then(response => response.json())
  .then(data => {
    // do something with the data here
  });

\end{lstlisting}

In this example, the headers object contains the Authorization header with a bearer token. This object is passed as the second argument to the fetch function, which includes the headers in the HTTP request.

\subsubsection{Using the abort controller}
The AbortController is a new API that allows you to abort an ongoing fetch() request. It is typically used when you want to cancel a request if the user navigates away from the current page, or if the user has started a new request that replaces the previous one.

Here is an example of how to use the AbortController with the fetch() method:

\begin{lstlisting}[language=Javascript, caption=AbortController example]
const controller = new AbortController();
const signal = controller.signal;

fetch('https://pokeapi.co/api/v2/pokemon/1', { signal })
  .then(response => response.json())
  .then(data => {
    console.log(data);
    // do something with the data here
  });

// later, if you want to cancel the request:
controller.abort();

\end{lstlisting}

\subsubsection{Fetch vs AJAX}
The main difference between fetch() and AJAX (Asynchronous JavaScript and XML) is that fetch() is a modern browser API, while AJAX is a technique used to send HTTP requests and retrieve data from a server. AJAX is based on the older XMLHttpRequest (XHR) API, which is supported by all modern browsers, but it has been largely replaced by the newer fetch() API.

Here are some other key differences between fetch() and AJAX:
\begin{itemize}
\item fetch() uses promises, while AJAX uses callbacks. This means that fetch() is easier to use and allows for more readable code, especially when dealing with asynchronous operations.
\item fetch() supports the streaming of data, which means that you can start processing the data as soon as it becomes available, rather than having to wait for the entire response to be received. AJAX does not support streaming.
\item fetch() supports the use of request and response objects, which provide a more powerful and flexible API for making web requests and handling responses. AJAX does not have this concept.
\end{itemize}
Overall, fetch() is a more modern and powerful API for making web requests, and it is the recommended way to perform HTTP requests in JavaScript.


\section{Other ways to use fetch}

The fetch function can be used for web scraping, but it is generally not the best option for this purpose. The fetch function is intended for making HTTP requests and retrieving data from a server, not for extracting data from an HTML page.

There are many dedicated tools and libraries that are better suited for web scraping, such as Puppeteer and Cheerio. These tools provide a more convenient and efficient way to extract data from HTML pages and can be easily integrated with the fetch function.

Here is an example of how you might use the fetch function and Cheerio to scrape data from an HTML page:

\begin{lstlisting}[language=Javascript, caption=Grabbing raw html with fetch and feeding that into a library]
fetch('https://example.com')
  .then(response => response.text())
  .then(html => {
    const $ = cheerio.load(html);
    const data = $('#some-element').text();
    // do something with the data here
  });

\end{lstlisting}

In this example, the fetch function is used to make a GET request to the example website,
and then the response is passed to the then callback function. The response.text() method is used to convert the response to a string of HTML, which is then passed to Cheerio's load method. This creates a Cheerio object that can be used to extract data from the HTML using jQuery like syntax. In this case, the \#some-element element is selected and its text content is extracted and stored in the data variable. From there, you can use the data however you like.

Again, this is just one example of how you might use the fetch function for web scraping. There are many other ways to accomplish this, and the specific approach you choose will depend on your specific needs and requirements.

\subsection{Introduction to caching}
A cache is a way of storing data so that future requests for the same data can be served faster. One way to use a cache with the fetch function (which is used to request data from a server) is to store the responses from fetch in a cache. Then, when a request is made for the same data, it can be served from the cache instead of making a new request to the server. This can improve the performance of your application by reducing the number of requests that need to be made to the server.

When data is requested from a server, it can be stored in a cache so that future requests for the same data can be served faster. This is because the data can be served from the cache instead of making a new request to the server. This can improve the performance of the application by reducing the amount of time it takes to serve data to the user.

Caching can be especially beneficial in applications that make many requests to the same server, or in applications that are used by a large number of users who may be requesting the same data. In these cases, caching can reduce the load on the server and improve the overall performance of the application.


Here is an example of using a simple cache with the fetch function:
\begin{lstlisting}[language=Javascript]
// Create a cache to store the responses from fetch
const cache = new Map();

// Define a function that uses fetch to request data from a server
function getData(url) {
  // Check if the data is already in the cache
  if (cache.has(url)) {
    // If it is, return the data from the cache
    return cache.get(url);
  } else {
    // If it's not in the cache, use fetch to request the data from the server
    return fetch(url)
      .then(response => response.json())
      .then(data => {
        // Store the data in the cache for future use
        cache.set(url, data);
        // Return the data
        return data;
      });
  }
}
\end{lstlisting}

In this example, the getData function uses fetch to request data from a server. If the data has been requested before, it will be served from the cache instead of making a new request to the server. This can improve the performance of your application by reducing the number of requests that need to be made to the server.


There are several JavaScript libraries that can be used to implement caching on the client side. Some examples include:
\begin{itemize}
\item lscache: This library is a simple in-memory cache that can be used to store data in the client's browser. It has a simple API and can be easily integrated into an application.
\item Memoizee: This library is a simple utility that can be used to memoize (cache) the results of expensive function calls. It can be used to improve the performance of an application by storing the results of frequently-used functions in a cache.
\item QuickLRU: This library is a simple, lightweight, and efficient LRU (Least Recently Used) cache. It can be used to store data in a cache and automatically remove the least recently used items when the cache reaches its maximum size.
\item tiny-lru: This library is a small and efficient LRU cache that can be used to store data in a cache. It has a simple API and is easy to integrate into an application.
\end{itemize}
These are just a few examples of JavaScript libraries that can be used to implement caching on the client side. There are many other libraries available, and the best one to use will depend on the specific requirements of your application.

There are more complete solutions like react-query that handle refreshing data, updating data as well as caching it.

To test your knowledge of async programming try to answer the following questions

\label{ex:3}
\begin{enumerate}
\item Implement a post webhook to a discord channel
\label{ex:4}
\item Parse the response from the pokeapi and return the url to the sprite of the master ball.
\end{enumerate}

See page \pageref{sol:3} for answers
}\cleanalldata

